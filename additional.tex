Description of Example Code

In order to illustrate the complex-step technique in a simpler form than it
appears in the \emph{Peridigm} code, an example header-only library and two
example problem definition programs that use it have been created. All the
materials described here are available in the associated project repository, 
or if this thesis is being viewed in an OCR processed electronic format, 
it may be convenient to copy the source-code from the appendix.

The source code for the header-only library, \emph{NewtonRaphson.hpp} is
reproduced in the appendix. Note that the library is an interface to functions
and data structures made available in the \emph{Trilinos} library, mentioned
earlier, such that it is required in order for the examples described here to
work. \emph{Trilinos} is freely available for download as of this writing at:
\emph{trilinos.sandia.gov}. Generally the header-only library works by taking
in the users initial guessed equilibrium solution, target dependent variable
values, system constants, and a function pointer to a model evaluation
function. Solver parameters including an update scaling factor, tolerance,
maximum iterations and probe length are chosen after the solver is
instantiated as a new object. The behavior of the solver is specialized through
a coding feature called inheritance to perform each of CS, AD and FD.  But, in
common, each version of the solver works by iteratively computing the system
Jacobian, solving a linear system as in that step of the Newton Raphson method,
computing the residual and checking for convergence or overrun.  Additionally
each specialization allows the problem to be fully or partially respecified to
allow for running a series of load steps having progressing target values,
which is a typical mode also seen in \emph{Peridigm}. Applying boundary
conditions incrementally in load steps assists convergence [PAPER].

The first example program, \emph{NRexamples.cpp}, solves a simple nonlinear
system describing the intersection of three infinite paraboloids in threespace.
It works by defining force computation functions for each of the models, and
then passing an address called a function pointer corresponding to each of
these methods to a solver object which coordinates solving the problem and
reporting output for the user based on their convergence criteria and other
preferences like step-size and maximum number of iterations. CS, AD and FD are
each represented and the user is free to adjust input settings by specifying
command line arguments as the compiled program instructs, or perhaps create a
shell script that varies parameters for each of the methods in order to perform
a comparative study.

The second program, \emph{NRComparison.cpp}, reproduces a case study that
appears in [PAPER]. The example program departs from the cited literature by
applying each of CS, AD and FD and verifying them against an analytical model
of a nonlinear mechanical system involving a connected beam and spring. The
equations describing this system and results for verification were obtained in
[PAPER]. The methods were used to solve the same system with the same load
steps as in the reference. Results indicate the each of the methods are capable
of reproducing the experimental results of the reference, when step-size is
chosen appropriately, however for small step-size the convergence rate of CS
superior to that of FD and equal to that resulting from relying on AD or an
analytically computed Jacobian. For extremely small step-size FD fails to
converge due to inaccurate Jacobian calculations in this example problem.
