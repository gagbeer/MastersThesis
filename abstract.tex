\documentclass[12pt]{report}
\begin{document} 
\centerline{A COMPARISON OF DIFFERENT METHODS FOR CALCULATING} 
\centerline{TANGENT-STIFFNESS MATRICES IN A MASSIVELY} 
\centerline{PARALLEL COMPUTATIONAL} 
\centerline{PERIDYNAMICS CODE} 
\noindent
\\ \\ \textbf{Name:} Michael Brothers \\ \\
%
\textbf{Supervising professor:} John T. Foster, Ph.D., Assistant Professor \\ \\
%
\textbf{Abstract:} \\ 
In order to maintain the quadratic convergence properties of Newton's method in
quasi-static nonlinear analysis of solid structures it is crucial to obtain
accurate, algorithmically consistent tangent-stiffness matrices. A goal of the
study described in this thesis was to establish the suitability of an
under-explored method for numerical computation of tangent-stiffness operators,
referred to as ``complex-step'', and compare the new method with other
techniques for numerical derivative calculation: automatic differentiation,
forward finite-difference, and central finite-difference. The complex-step
method was newly implemented in a massively parallel computational peridynamics
code for the purpose of this comparison. The methods were compared through in
situ profiling of the code for accuracy, speed, efficiency, and parallel
scalability. The research provides data that can serve as practical guide for
code developers and analysts faced with choosing which method best suits the
needs of their application code. Additionally, motivated by the reproducible
research movement, all the of the code, examples, and workflow to regenerate
the data and figures in this thesis are provided as open source. \\ \\
%
\textbf{Date of defense:}  November 22, 2013 \\ \\
% 
\textbf{Time:} 3:00 pm \\ \\
%
\textbf{Place:} ME Conference Room EB 3.04.62 \\ \\
%
\textbf{Campus:} 1604 \\ \\
\end{document}
